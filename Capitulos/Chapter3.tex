\chapter{Bases de Datos Espacio-temporales}  \label{cap:et}

\section{Introducción} \label{sec:st:intro}

En este capitulo aplicaremos un enfoque distintos a los capitulos anteriores.
Para el estudio de las bases de espaciales y temporales fue necesario presentar varios conceptos teóricos,
la mayoría provenientes de otras ramas científicas,
que construyen los cimientos para la implementación de las mismas.
Luego, presentamos los estándares ya afirmados en cada área,
justificando con estos marcos teóricos las decisiones de diseño que se tomo en cada uno.

En el campo de las \concept{bases de datos espacio-temporales},
que es como llamaremos a las bases de datos que combinan de alguna forma ambas funcionalidades,
veremos que el panorama es un poco distinto.
Debido a la gran amplitud en requerimientos de las distintas aplicaciones,
existen muchos y variados modelos, cada uno caracterizado por
qué elementos toma de las funcionalidades espaciales, cuáles de las temporales y como los combina.
La mayoría de estos modelos se focalizan en una o pocas aplicaciones.
Si bien existen modelos mas generales, son muy complejos conceptualmente
y no muy fáciles de optimizar para problemas específicos.

Por lo tanto, a lo largo de este capítulo iremos presentando brevemente, en orden incremental de complejidad,
diferentes modelos y sus aplicaciones, intercalándolos con conceptos y definiciones cuando sea necesario.
Para un estudio comparativo mas en profundidad de los distintos modelos, sugerimos recurrir a \cite{sp:litreview}.

\section{Modelo de Snapshots}

% Propuesta
Comenzaremos por el \concept{modelo de snapshots}\textsuperscript{\cite{sp:snapshot}}, por ser uno de los mas sencillos conceptualmente.
Este modelo divide los datos en capas temporalmente homogéneas, con un timestamp único.
De esta manera, muestra el estado de la distribución espacial en diferentes tiempos sin relaciones explícitas entre capas.

% Ventajas
Es un modelo que si bien representa aspectos espaciales y temporales,
ambos se encuentran muy separados y no agrega comportamientos complejos.
Desde el punto de vista espacial,
podemos pensar el tiempo como un atributo no espacial (en el sentido definido en la sección \ref{subsec:mbo}).
Desde el punto de vista temporal,
a las consultas espaciales podemos agregarle condiciones de tiempo de validez similares a las que vimos en la sección \ref{subsec:pau}.

% Desventajas
Si bien como una primera aproximación es sencilla e intuitiva, la división en capas que propone este modelo tiene varias desventajas.
\begin{itemize}
    \item El modelo no es apropiado para describir cambios en el espacio a través del tiempo.
        Es difícil determinar cambios entre dos momentos ya que hay que comparar los snapshots.
    \item Todo cambio produce una copia completa en cada porción de tiempo,
        generando una importante duplicación de datos.
        Este problema se ve incrementado por la tendencia de los datos espaciales de por sí a ser muy pesados.
    \item Es muy difícil diseñar o aplicar reglas para la lógica interna o la integridad.
        El modelo no proporciona una comprensión de las restricciones en la estructura temporal.
\end{itemize}


\section{MADS}

% Propuesta
El modelo de snapshots y muchos otros modelos posteriores desarrollaron la idea de representar datos espaciales que pueden cambiar.
Pero un enfoque diferente, propuesto por primera vez por \cite{CPS98},
propone \textit{representar los procesos de cambio que actuan sobre los atributos geométricos de una entidad}.

Un importante exponente de este modelo es \concept{MADS}\textsuperscript{\cite{PSZ99}}
(sigla en inglés de "Modeling Application Data with Spatio-temporal features").
MADS se propone incorporar los conceptos básicos de modelado de espacio y tiempo en el modelo objeto-relación.
Los procesos son representados como relaciones entre objetos espacio-temporales.

% TODO: mejorar

\begin{verbatim}
    OBJECT Parcel
        TEMPORAL Day
        GEOMETRY AREA TEMPORAL DAY
        ATTRIBUTES
            Number: INTEGER
            Name: string [1:N] TEMPORAL DAY
            Owner: [1:N] TEMPORAL DAY
    END Parcel
\end{verbatim}


\section{MOST}

% Objetos en Movimiento
El último paradigma que veremos es el de los modelos que se centran en representar objetos en movimiento.
