\chapter{Bases de Datos Espaciotemporales}  \label{cap:et}

\section{Introduccion}
\label{sec:1}
Comparacion de las dos culturas de investigacion que ya discutimos para introducir que hay varias tendencias en el campo.

\section{El tiempo como una cuarta dimension}
\label{sec:2}
En esta seccion hablamos de lo intuitivo que es combinar ambos problemas pero a la vez las complicaciones que trae.

\section{Conceptos}
\label{sec:3}
Use the template \emph{chapter.tex} together with the Springer document class SVMono (monograph-type books) or SVMult (edited books) to style the various elements of your chapter content in the Springer layout.

\section{Queries espacio temporales}
\label{sec:4}
Mencionamos por arriba queries propias de espacio temporales que podrian ser interesantes. Algo estilo "ademas de las combinaciones obvias, hay algunas muy propias de espaciotemporales:"

\section{Algunos modelos}
\label{sec:5}
Los mismos de las slides. La relfexion de que parte de diseñar una BDET es elegir el modelo que mas te sirva para el caso.

% \input{referenc}