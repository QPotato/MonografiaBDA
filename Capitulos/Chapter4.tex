\chapter{Estado del Arte y Conclusiones}  \label{cap:conc}

A lo largo de este trabajo, hemos estudiado y comparado los distintos modelos que han surgido en las últimas décadas para representar datos espaciales, temporales y espacio temporales. Para terminar, presentaremos una visión general del estado del arte en estas ramas y nuestras conclusiones sobre como pueden desarrollarse a futuro.

Si bien las bases de datos datos espaciales y temporales tienen ya varias décadas de historia como ramas de investigación, sumadas a las de de sus precursoras (geografía y lógica temporal respectivamente), que resultaron en la estandarización de modelos y sus respectivas extensiones de SQL. Las bases de datos espaciotemporales, por su parte, solo recientemente se conformaron como una rama propia y se encuentra actualmente fragmentada en modelos optimizados para aplicaciones específicas.

Aun así, las primeras han perdido interés propio mientras que la última acapara cada vez más atención. Actualmente, es poco común encontrar grupos de investigación exclusivamente de bases de datos espaciales y menos aun de temporales. Aquellos grupos que empezaron focalizados en uno solo de estos aspectos de los datos, en general han girado su interés a la ampliación de sus descubrimientos al campo espaciotemporal.

Así, el grupo dirigido por Shashi Shekhar de la Universidad de Minesota se destaca por sus recientes aportes en minería de datos espacial\textsuperscript{\cite{spatial:datamining}} y, mas recientemente, minería de datos espaciotemporal\textsuperscript{\cite{spatiotemporal:datamining}}. También Markus Schenider, a quién mencionamos en la sección \ref{realms:1} como uno de los proponentes del Álgebra de RoSE, publicó aportes sobre tipos de datos en modelos de objetos en movimiento\textsuperscript{\cite{schenider:moving}} y cálculo de datos espaciotemporales\textsuperscript{\cite{schenider:calculo}}. Por su parte, el grupo de la Universidad de Lancaster dedicado a la ciencia de datos geoespacial se encuentra estudiando el modelado de procesos de cambio climático usando bases de datos espaciotemporales\textsuperscript{\cite{climate}}.

Por lado de las temporal, el grupo DDCM (Database, Document and Content Management) de la Universidad
de Gante, Bélgica generalizo a espaciotemporales\textsuperscript{\cite{gant:st}} su investigación sobre datos
temporales con incerteza\textsuperscript{\cite{gant:t}}. El Centro de Información e Ingeniería de Bases de Datos de la Universidad de Zurich, investigó diferentes beneficios que puede tener usar al recta temporal como
índice universal, en un modo análogo a los realms en el Álgebra de RoSE y similares. Por ejemplo en \cite{sweeping}, se usa para definir operaciones de agregado temporales eficientes.

Los autores de este trabajo consideramos que el estado actual de fragmentación en los modelos de bases de datos espaciotemporales no es una tendencia que continuará a largo plazo. Tanto las bases de datos espaciales como las temporales requirieron de décadas de estudios y avances que permitieron comparar distintos enfoques y generalizar los más prometedores sin degradar su desempeño en los casos de aplicación de mayor interés, pero dejando aquellos que si bien tenían su importancia en sus ramas de investigación precursora, no resultan tan interesantes en el modelado de bases de datos (como los espacios continuos o el tiempo ramificado). Creemos que este fenómeno se replicará eventualmente en el estudio de bases de datos espaciotemporales y si esto parece distante hoy, es no solo por la complejidad de los problemas estudiados sino también por las dificultades que trajo al campo la unión de dos culturas de investigación tan distintas como son la geografía y la lógica temporal.