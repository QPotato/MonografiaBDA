\newcommand{\concept}{\textbf}

\chapter{Bases de Datos Espaciales}  \label{cap:e}

\section{Introducción}

% Que es una base de datos espacial
Comenzaremos este trabajo con el estudio de los datos espaciales. Llamamos \concept{bases de datos espaciales} a aquellas que incorporan capacidades de representación y manipulación de datos geométricos relativos a un sistema de referencia. Llamaremos también \concept{SDBMS} (Spatial Database Managent System) a los motores de bases de datos que incorporan estas funcionalidades.

% GIS
Su origen y principal aplicación son los Sistemas de Información Geográfica (GIS): software que provee mecanismos de análisis y visualización de de datos geográficos, es decir, datos espaciales cuyo sistema de referencia  es la superficie terrestre.

Los GIS implementan en sus herramientas un gran conjunto de técnicas desarrolladas por cartógrafos a lo largo de la historia y que son previas al desarrollo de las tecnologías informáticas. Este hecho dota a la investigación de las bases de datos espaciales de un carácter multidisciplinario que puede señalarse como una de las causas del rápido avance de la misma.

% Otras aplicaciones
A su vez, existen otras aplicaciones para las bases de datos espaciales tales como la representación de circuitos, datos astronómicos, moléculas y muchas otras.
En general, los modelos teóricos para estas aplicaciones son los mismos, siendo la única diferencia simplemente el sistema de referencia en el que se representan los datos.
En este trabajo haremos foco en las aplicaciones a los GIS, pero el lector no debe perder de vista que todos los conceptos discutidos son aplicables a estas otras áreas.

% Org del capitulo
En la seccion \ref{sec:s:models} estudiaremos los distintos modelos teóricos existentes en el campo de las bases de datos espaciales . Luego, en la seccion \ref{sec:s:sdt} nos enfocaremos en un paradigma particular para ver mas en detalle los tipos de datos que incorpora y algunas de sus operaciones.

\section{Modelos Teóricos} \label{sec:s:models}

% Contexto
El desarrollo de modelos teóricos para las bases de datos espaciales debe entenderse en el contexto histórico discutido en la sección anterior.
Los primeros GIS implementaban todas las operaciones sobre datos espaciales a nivel de aplicación, guardando sus datos en bases de datos convencionales.
A su vez, los primeros motores de bases de datos espaciales intentaron simplemente mover esta lógica al nivel de base de datos, pero sin desarrollar un marco teórico.
Posteriormente, algunos investigadores comenzaron a dar definiciones formales de \concept{tipos de datos espaciales}.

% Org de la seccion
%TODO: rehacer esto al final
Comenzaremos analizando en las secciones \ref{realms:1} y \ref{realms:2} el Álgebra de RoSE\textsuperscript{\cite{rose}} (Robust Spatial Extension), un modelo formal que consideramos uno de los más elegantes y significativos del campo.
Luego, en las secciones \ref{sec:s:fun} y \ref{s:obj} veremos los principales paradigmas de modelado que se utilizan en los sistemas de bases de bases de datos espaciales actuales, los modelos funcionales y los orientados a objetos.

\subsection{Realms} \label{realms:1}

Según Güting y Schenider\textsuperscript{\cite{rose}}, un modelo formal para Bases de Datos Espaciales debe ser:
\begin{itemize}
    \item General: los objetos geométricos usados como valores de SDTs deben ser tan generales como sea posible.
        Por ejemplo, un valor región debe poder representar una colección de áreas disjuntas, cada una de las cuales puede tener agujeros.
        Mas precisamente, los dominios de los tipos de datos punto, línea y región deben ser cerrados respecto a la unión,
        intersección y diferencia de sus conjuntos de puntos subyacentes.
    \item Riguroso: la semántica de los SDTs, es decir, los posibles valores para los tipos y las funciones asociadas con las operaciones,
        deben estar definidas formalmente para evitar ambigüedades para el usuario y el implementador.
    \item De resolución finita: las definiciones formales deben tener en cuenta las capacidades de representación finitas de las computadoras
        Delegarle al programador la responsabilidad de cerrar la brecha entre teoría y práctica en este punto lleva a errores numéricos y topológicos.
    \item Geométricamente consistente: distintos objetos espaciales pueden estar pueden estar relacionados mediante restricciones geométricas.
        Las definiciones de los SDTs ayudar a mantener esa consistencia.
\end{itemize}

% Resumen
El Álgebra de RoSE se basa en la incorporación a los DMBSs del concepto de realm, un conjunto finito de puntos y segmentos sin intersecciones
sobre los cuales se posicionan todos los datos espaciales de una base de datos.
Con este enfoque, los datos no se crean al darle valor a un atributo, sino que se \emph{seleccionan} del conjunto de
valores existentes en el realm.
Este modelo permite asegurar que el álgebra espacial es cerrada respecto a un realm.
Es decir, que la primitivas geométricas y las operaciones sobre realms se definen sobre
aritmética entera, libre de errores de redondeo que podrían aparecer si el modelo se definiera sobre un espacio euclídeo.

% Computo complejo en los inserts y updates, no en busquedas.
Otra ventaja de los realms es que permiten aislar todo el computo de puntos en las operaciones de actualización del realm.
No se computan intersecciones durante consultas de búsqueda.

% Finite resolution
Para mapear los segmentos con intersecciones de una aplicación a un conjunto de segmentos sin intersecciones de un realm,
se utiliza redibujado y geometría de resolución finita\textsuperscript{\cite{finite:resolution}}.

\subsection{Actualizaciones en realms} \label{realms:2}

% Redibujado
Veamos mas en detalle que ocurre en este modelo cuando se realiza una actualización que requiere la insercion de un nuevo segmento.
Esto es importante porque los datos geométricos provenientes de la aplicación no son no-intersecantes como requiere el modelo.
El problema fundamental es que usualmente los puntos de intersección no se corresponden con puntos de la grilla del realm.
La solución es aplicar redibujado:
se modifican los segmentos de forma tal que la intersección se mueva al punto mas cercano que sí esté en la grilla.

% Ejemplo de redibujado
Esto puede verse en las figuras \ref{fig:realms:redibu:a} y \ref{fig:realms:redibu:b}:
la intersección entre los segmentos A y B (D') no se corresponde con un punto de la grilla,
pero podemos partir a cada uno de ellos en dos, de manera que todos los segmentos resultantes tengan a Dcomo uno de sus extremos.

\begin{figure}
    \centering
    \begin{subfigure}[b]{0.3\textwidth}
        \includegraphics[width=\textwidth]{fig-realms-a.png}
        \caption{}
        \label{fig:realms:redibu:a}
    \end{subfigure}
    \begin{subfigure}[b]{0.3\textwidth}
        \includegraphics[width=\textwidth]{fig-realms-b.png}
        \caption{}
        \label{fig:realms:redibu:b}
    \end{subfigure}
    \caption{Redibujado de la intersección de dos segmentos.}
    \label{fig:realms:redibu}
\end{figure}

% Envolturas
Pero ahora nos surge un nuevo problema que es la preocupación de que aplicando esta operación sucesivamente
los segmentos se modifiquen cada vez mas, introduciendo cada vez mas error en la representacion.
Podemos acotar este error que se introduce, usando el concepto de envoltura.
La envoltura de un segmento es el conjunto de puntos de la grilla que se encuentran inmediatamente arriba, debajo o sobre el segmento.
Si establecemos entonces la restriccion de que los puntos a usar para el redibujado deben caer sobre la envoltura del segmento,
tenemos una cota para el error: la distancia entre puntos en la grilla.
La figura \ref{fig:realms:envoltura} muestra un ejemplo de envoltura de un segmento.

\begin{figure}
    \centering
    \includegraphics[width=\textwidth]{fig-realms_envoltura.png}
    \caption{Envoltura de un segmento.}
    \label{fig:realms:envoltura}
\end{figure}

% Conclusion
De esta forma, Güting y Schneider definen un conjunto de operaciones geométricas primitivas robustas con error acotado que son la base para,
a través de la composición de las mismas, definir de forma rigurosa todos los SDTs
y las operaciones geométricas que nos pueden resultar interesantes para la representación de datos espaciales.
Si bien el Álgebra de RoSE fue un hito en términos de rigurosidad teórica en los modelos de bases de datos espaciales,
no es utilizada actualmente en ninguno de los SDBMSs actualmente en uso por la industria del software.
De todas formas, su influencia puede verse en los modelos que se utilizan actualmente y que veremos a continuación.

\subsection{Modelo Funcional} \label{sec:s:fun}

% Intro y ejemplo
En esta sección, veremos el modelo funcional o modelo orientado a campos.
Comenzaremos presentando un ejemplo de un problema espacial a modelar. Supongamos un parque nacional que cuenta con
un conjunto de bosques, cada uno con una especie de árbol dominante.

% Ejemplo en funcional
En el modelo funcional, se pueden modelar los bosques del parque nacional como
una función cuyo dominio es el conjunto de puntos en el mapa del parque nacional y el rango es un conjunto de especies de árboles.
En este caso, se trata de una funcion escalonada (constante dentro en un mismo bosque y con saltos en los bordes entre bosques).


% Componentes de un modelo funcional
En general, para modelar un problema usando el paradigma funcional, se deben definir tres componentes\textsuperscript{\cite{worboyz}}:
\begin{itemize}
    \item Un sistema de referencia espacial: una grilla finita sobre el espacio subyacente,
    similar al concepto que presentamos en la sección \ref{realms:1} sobre el Álgebra de RoSE.
    El ejemplo mas conocido es un sistema de referencia de la tierra por latitud y longitud.
    \item Campos simples: un conjunto finito de funciones computables $\{f_i: SF \rightarrow A_i, 1 \le i \le n\}$
    donde SF es el sistema de referencia espacial y los $A_i$ son dominios de atributos. La elecciones de estos dominios de atributos
    y las funciones que asignan elementos de los mismos a cada punto del espacio dependerá de la aplicación en cuestión.
    \item Operaciones de campos: especifican las relaciones e interacciones entre los distintos campos. Estas se dividen en:
    \begin{itemize}
        \item Operaciones locales: operaciones en las que el valor del campo resultante solo depende de los valores de los campos de entrada en ese punto.
        La composición de campos es un ejemplo de operación local.
        \item Operaciones focales: operaciones en las que el valor del campo resultante depende de los valores de los campos en un pequeña vecindad alrededor del punto.
        Este es el caso por ejemplo de operaciones que involucran gradientes.
        \item Operaciones zonales: operaciones que involucran operadores agregados como el promedio o la integración. También son operaciones
        zonales aquellas que particionan el espacio en zonas.
    \end{itemize}
\end{itemize}

% Ventajas de funcional
Los modelos funcionales son buenos para representar problemas en los que las datos continuos
como elevación, temperatura y variación del suelo.
Tienen la ventaja de poder representar fenómenos espaciales amorfos o con contornos fluidos.

\subsection{Modelos Basados en Objetos} \label{sec:s:obj}

%TODO: \cite{oo:spatial}
% Ejemplo en objetos
Volviendo al ejemplo del parque nacional, si consideramos los lugares donde los bosques cambian,
en un entorno idealizado donde los bordes entre bosques están claramente definidos, obtendremos los bordes de polígonos.
A cada polígono se le puede asignar un identificador y un atributo no espacial (la especie de árbol dominante).
Los bosques del parque pueden modelarse entonces como un conjunto de polígonos.

% definicion objetos
En el modelado basado en objetos, se abstrae información espacial en conjuntos de
entidades únicas, distinguibles y relevantes llamadas objetos.
Cada objeto tiene un conjunto de atributos que lo caracterizan, los cuales se dividen en atributos espaciales y no espaciales.
En el ejemplo del parque nacional, un objeto bosque tiene un atributo espacial, un polígono, que representa su extensión espacial,
y un atributo no espacial, su espacie dominante, representada como un valor alfanumérico.
Es interesante notar que un objeto puede tener varios atributos espaciales.
Por ejemplo, un camino puede tener un polígono y una línea, para elegir cuál usar según la escala del mapa.

% Ventajas objetos
Los modelos de objetos son los mas comunes en problemas relacionados a representar redes de transportes o parcelas de tierra.
Tienen la ventaja de ser mas intuitivos y cercanos a los modelos Entidad-Relación ya muy presentes en los DBMS más usados.
En definitiva, la decisión de usar el paradigma funcional o de en objetos se basará en los requerimientos de la aplicación.

\section{Tipos de Datos y Operaciones Espaciales} \label{sec:s:sdt}

En esta sección analizaremos los tipos de datos que introducen las bases de datos espaciales, enfocándonos en el paradigma basado en objetos. Veremos también algunos ejemplos de operaciones comunes con ellos.

\subsection{Tipos de Datos Espaciales}

Se han propuesto muchos conjuntos básicos de tipos espaciales para la representación de formas comunes en mapas. Actualmente, la más adoptada es el estándar OGIS\textsuperscript{\cite{99opengis}} y es en la que nos basaremos.

El estándar OGIS propone los siguientes tipos de datos:
\begin{itemize}
    \item Geometría: es un tipo abstracto (no puede ser instanciado) del cual se derivan todos los otros tipos.
    Tiene asociado un sistema de referencia.
    \item Punto: describe la posición de objetos de cero dimensiones,
    como una oficina o una zona de campamento, en el caso del parque nacional.
    \item Cadena de Lineas: describe un objeto de una dimensión a través de una sucesión de de dos o mas puntos,
    que pueden representar un segmento o aproximar una curva. Por ejemplo, caminos o ríos.
    \item Polígono: objetos de 2 dimensiones, como un bosque.
    \item Colección de Geometrías: formas complejas formadas por conjuntos de otros tipos.
    Se dividen en multipuntos, multilineas y multipoligonos.
    Estos tipos son necesarios para asegurar que las operaciones geométricas sean cerradas.
    Por ejemplo, la intersección de un río con un bosque de forma cóncava pueden ser representado por un conjunto de líneas.
\end{itemize}

\subsection{Operaciones Espaciales}

% Taxonnomía
En el modelo funcional, las relaciones disponibles quedaban definidas por las funciones de campos del modelo.
Pero en un modelo basado en objetos, el espacio subyacente será el que determine las operaciones disponsibles
y las relaciones que pueden existir entre objetos. La siguiente tipología de espacios resume la presentada por \cite{spatial:book}:

\begin{itemize}
    \item Espacios Orientados a Conjuntos: son los espacios mas simples y generales.
    Permiten todas las relaciones usuales de conjuntos como unión, intersección, contención y pertenencia.
    \item Espacios Topológicos: los espacios topológicos son aquellos en los que las relaciones no se ven afectadas
    por transformaciones elásticas del mismo.
    \item Espacios Direccionales: las reaciones direccionales de un espacio pueden ser
    absolutas (en referencia a un sistema de referencia global),
    relativas (basadas en la orientación de un objeto)
    o basadas en un observador (un objeto especial designado como tal).
    \item Espacios Métricos: son aquellos espacios en los que la noción de distancia está bien definida. La función de distancia puede usarse
    para definir una topología y por lo tanto todo espacio métrico es también un espacio topológico.
    \item Espacios Euclidianos: Son espacios en los que los puntos pueden expresarse como la combinación lineal de los elementos de
    un sistema de referencia. Sobre espacios euclidianos pueden definirse todas las operaciones de las categorías anteriores.
\end{itemize}

%TODO: Ejemplos de tablas espaciales
Teniendo en cuenta esto, veamos ahora algunos ejemplos de las operaciones que define el estándar OGIS.
Consideremos una base de datos espacial con las siguientes tablas que definen países, ciudades y ríos.

\begin{verbatim}
    CREATE TABLE Pais(
        nombre varchar(30),
        continente varcahr(30),
        pbi integer,
        geometria Polygon
    );

    CREATE TABLE Ciudad(
        nombre varchar(30),
        pais varchar(30),
        poblacion integer,
        geometria Point
    );

    CREATE TABLE Rio(
        nombre varchar(30),
        largo Number,
        geometria LineString
    );
\end{verbatim}

Podemos usar el predicado \texttt{Touch} para encontrar los países limítrofes de Argentina.

\begin{verbatim}
    SELECT P1.nombre AS "Limítrofes de Argentina"
    FROM Pais P1, Pais P2
    WHERE Touch(P1.geometria, P2.geometria) = 1
          AND P2.nombre = 'Argentina';
\end{verbatim}

Podemos usar el predicado \texttt{Cross} para encontrar todos los paises por los cuales pasa cada río.

\begin{verbatim}
    SELECT R.nombre, P.nombre
    FROM Rio R, Pais P
    WHERE Cross(R.forma, P.forma) = 1
\end{verbatim}

Por último veamos una consulta mas compleja que combina dos operaciones nuevas. Supongamos que el Río Paraná puede proveer de agua a todas las ciudades a menos de 300km de distancia. Primero, podemos usar \texttt{Buffer} para encontrar el área a 300km de dicho río (esto es una generalización de las envolturas que vimos en la sección \ref{realms:2}) y luego, \texttt{Overlap} para encontrar aquellas ciudades que tienen intersección con esta área.

\begin{verbatim}
    SELECT C.nombre
    FROM Ciudad C, Rio R
    WHERE Rio.nombre = "Paraná"
          AND Overlap(C.forma Buffer(R.forma, 300)) = 1
\end{verbatim}

Para un listado completo de los operadores y predicados espaciales incorporados al estándar SQL, sugerimos referirse a \cite{99opengis}.
%TODO: una tabla de todas las operaciones espaciales, robada de algun lado.

% \section{Procesamiento de consultas espaciales}

% % Presentacion de filter-refine
% En esta sección exploraremos una técnica para la optimización de consultas espaciales sumamente importante:
% el paradigma de filtrado y refinamiento.

% % Consultas declarativas
% Los usuarios expresan las consultas a un motor de bases de datos utilizando un lenguaje declarativo como SQL.
% Esto quiere decir que el usuario solo define qué resultado quiere, pero es responsabildiad del motor de bases de datos
% determinar cuál es el proceso más eficiente para llegar a ese resutado y luego ejecutarlo. A esto le llamamos procesamiento de consultas.

% % Consultas single-scan vs multi-scan
% Las consultas a su vez, pueden dividirse en las de escaneo simple o escaneo múltiple.
% Es una consula de escaneo simple,cada registro de una tabla deberá ser accedido a lo sumo una vez durante el procesamiento de la consulta.
% La operación Join es el ejemplo típico de una operación de escaneo múltiple