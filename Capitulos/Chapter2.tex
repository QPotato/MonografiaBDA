\chapter{Bases de Datos Temporales}  \label{cap:t}


\section{Introduccion}
\label{sec:t1}
Introduccion a la problema de representar el tiempo en BDs haciendo enfasis en que primero fue teorico y despues se fueron encontrando usos interesantes.

\section{Tiempo de Validez vs Tiempo de Transaccion}
\label{sec:t2}
Tiempo de Validez vs Tiempo de Transaccion

\section{Queries Temporales}
\label{sec:t3}
Mencionamos, con ejemplo, algunas queries que pueden ser interesantes.

\section{Modelos matematicos}
\label{sec:t4}
Introduccion a la problema de representar el tiempo en BDs haciendo enfasis en que primero fue teorico y despues se fueron encontrando usos interesantes.

\section{Representacion de imprecision}
\label{sec:t5}
Comparado contra espaciales?

\section{Recta Temporal}
\label{sec:t6}
Aca introudcimos el concetp y cerramos concluyendo que termina siendo analogo a los Realms.

% \input{referenc}