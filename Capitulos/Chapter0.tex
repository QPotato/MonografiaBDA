\usepackage[utf8]{inputenc}
\chapter{Introducción} \label{cap:intro}

La idea general es presentar espaciales como algo practica->teoria, temporales como teoria->practica y despues espacio temporales como una mezcla de ambas "culturas".

Intro hablando sobre el marco general de la mono, detallando que va a tratar cada capitulo y como esta estructurada

En esta trabajo, se presentan conceptos generales y el estado actual de la investigación científica en el área de las Bases de Datos Espaciales, Temporales y Espacio-Temporales.

En el capitulo 1 se presentan conceptos sobre Bases de Datos Espaciales, antecedentes importantes en el procesamiento de datos espaciales como los sistemas GIS, los modelos teóricos que se usan actualmente, diversas aplicaciones y generalizaciones teóricas que nos serán útiles para combinarlas con modelos temporales.

En el capítulo 2 se presentan modelos de Bases de Datos Temporales, las distintas formas de modelarlas teóricamente, el consenso obtenido entre distintos modelos y posterior extensión al estándar SQL y se compara contra el proceso de desarrollo de las Espaciales.

En el capítulo 3 se examina como de la combinación de ambas ramas de investigación y desarrollo surgen las Bases de Datos Espacio-Temporales, las distintas tendencias heredadas de las diferencias historicas entre Temporales y Espaciales que derivan en una variedad de modelos especializados para aplicaciones especificas y se concluye con un análisis de como los autores de este trabajo ven la posible evolución de esta area