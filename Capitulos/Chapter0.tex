\chapter{Introducción} \label{cap:intro}

La representación y almacenamiento de datos espaciales, asi como la de los temporales son ambas ramas de investigación con décadas de historia que, en las últimas, tendieron a solaparse ante la evidencia de la similitud en la estructura de los problemas tratados. Esta unión necesitó del trabajo en equipo de geógrafos y matemáticos de la lógica temporal (protagonistas respectivamente de cada rama) y dio origen a una gran variedad de modelos de bases de datos espaciotemporales con adaptaciones y optimizaciones para sus distintos campos de aplicación, sin que exista hoy un modelo estándarizado que se use en todas las aplicaciones con datos espaciotemporales.

En este trabajo, estudiaremos conceptos generales y el estado actual de la investigación científica en bases de datos espaciales, temporales y espaciotemporales. Luego, presentaremos nuestras conclusiones sobre como puede evolucionar esta área en las próximas décadas y sobre la posibilidad de la convergencia en un modelo estándar de bases de datos espaciotemporales.

En el capítulo \ref{cap:e} presentaremos conceptos sobre bases de datos espaciales, antecedentes importantes en el procesamiento de datos espaciales como los sistemas de información geográgica, los modelos teóricos que se usan actualmente y sus diversas aplicaciones.

Luego, en el capítulo \ref{cap:t} estudiaremos modelos de bases de datos temporales, sus bases en la lógica temporal, las distintas formas de modelarlas teóricamente y el consenso obtenido entre distintos modelos con su posterior incorporación al estándar SQL.

Posteriormente, en el capítulo \ref{cap:et} veremos como surgen de la combinación de ambas ramas de investigación y desarrollo las bases de datos espaciotemporales, las distintas tendencias heredadas de las diferencias historicas entre temporales y espaciales y como estas derivan en una variedad de modelos especializados para aplicaciones especifícas.

Por último, en el capítulo \ref{cap:e} examinaremos el estado del arte de estas ramas de investigación y presentaremos el análisis de los autores de este trabajo sobre la posible evolución del área.
