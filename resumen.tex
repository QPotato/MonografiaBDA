\chapter{Resumen}

% El resumen de la monografía.
La representación y almacenamiento de datos espaciales, así como la de los temporales, son ambas ramas de investigación con décadas de historia que, en las últimas, tendieron a solaparse ante la evidencia de la similitud en la estructura de los problemas tratados. Esta unión necesitó del trabajo en equipo de geógrafos y matemáticos de la lógica temporal y dio origen a una gran variedad de modelos de bases de datos espaciotemporales con adaptaciones y optimizaciones para sus distintos campos de aplicación.

En esta trabajo, presentaremos conceptos generales y el estado actual de la investigación científica en bases de datos espaciales, temporales y espaciotemporales. Luego, presentaremos nuestras conclusiones sobre como puede evolucionar esta área en las próximas décadas.